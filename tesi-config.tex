%**************************************************************
% file contenente le impostazioni della tesi
%**************************************************************


%**************************************************************
% Impostazioni di impaginazione
% see: http://wwwcdf.pd.infn.it/AppuntiLinux/a2547.htm
%**************************************************************

\setlength{\parindent}{14pt}   % larghezza rientro della prima riga
\setlength{\parskip}{0pt}   % distanza tra i paragrafi



%**************************************************************
% Impostazioni di biblatex
%**************************************************************
\bibliography{bibliografia} % database di biblatex 

\defbibheading{bibliography} {
    \cleardoublepage
    \phantomsection 
    \addcontentsline{toc}{chapter}{\bibname}
    \chapter*{\bibname\markboth{\bibname}{\bibname}}
}

\setlength\bibitemsep{1.5\itemsep} % spazio tra entry

\DeclareBibliographyCategory{opere}
\DeclareBibliographyCategory{web}

\addtocategory{opere}{womak:lean-thinking}
\addtocategory{web}{site:agile-manifesto}

\defbibheading{opere}{\section*{Riferimenti bibliografici}}
\defbibheading{web}{\section*{Siti Web consultati}}


%**************************************************************
% Impostazioni di caption
%**************************************************************
\captionsetup{
    tableposition=top,
    figureposition=bottom,
    font=small,
    format=hang,
    labelfont=bf
}

%**************************************************************
% Impostazioni di glossaries
%**************************************************************

%**************************************************************
% Acronimi
%**************************************************************
\renewcommand{\acronymname}{Acronimi e abbreviazioni}

\newacronym[description={\glslink{apig}{Application Program Interface}}]
    {api}{API}{Application Program Interface}

\newacronym[description={\glslink{umlg}{Unified Modeling Language}}]
    {uml}{UML}{Unified Modeling Language}

\newacronym[description={\glslink{mvpg}{Model View Presenter}}]
    {mvp}{MVP}{Model View Presenter}

%**************************************************************
% Glossario
%**************************************************************
%\renewcommand{\glossaryname}{Glossario}
\renewcommand{\glsnamefont}[1]{\makefirstuc{#1}}


\newglossaryentry{branch}
{
    name=branch,
    sort=branch,
    description={Rappresenta una linea di sviluppo indipendente e serve come astrazione per il processo di modifica,
stage o commit.}
}
\newglossaryentry{fork}
{
    name=fork,
    sort=fork,
    description={Sviluppo di un nuovo progetto software che parte dal codice sorgente di un altro già esistente}
}

\newglossaryentry{pull request}
{
    name=pull request,
    sort=pull request,
    description={È una richiesta, fatta all’autore originale di un sofware o di un documento, di includere le nostre personali modifiche al suo progetto. Una pull request controlla automaticamente se sono presenti conflitti tra i file da unire.}
}

\newglossaryentry{agile}
{
    name=agile,
    sort=agile,
    description={È un modello di sviluppo che si contrappone ai classici modelli a cascata o incrementali proponendo un approccio meno strutturato e focalizzato sull'obiettivo di consegnare al cliente, in tempi brevi e frequentemente, software funzionante e di qualità. }
}
\newglossaryentry{BMI}
{
    name=BMI,
    sort=bmi,
    description={Body Mass Index è un dato biometrico, espresso come rapporto tra peso e quadrato dell'altezza di un individuo ed è utilizzato come un indicatore dello stato di peso forma. }
}
\newglossaryentry{bot}
{
    name=bot,
    sort=bot,
    description={Con la parola “bot”, abbreviazione di “robot”, in informatica si intende un programma che ha accesso agli stessi sistemi di comunicazione e interazione con le macchine usate dagli esseri umani.}
}

\newglossaryentry{JWT}
{
    name=JWT,
    sort=jwt,
    description={JSON Web Token è uno standard Internet proposto per la creazione di dati con firma opzionale e/o crittografia opzionale il cui payload contiene un JSON che afferma un certo numero di attestazioni. I token vengono firmati utilizzando una crittografica simmetrica o asimmetrica utilizzando una coppia di chaivi pubblica/privata.}
}
\newglossaryentry{gamification}
{
    name=gamification,
    sort=gamification,
    description={È l'utilizzo di elementi mutuati dai giochi e delle tecniche di game design in contesti non ludici. L'idea è quella di coinvolgere le persone a provare più divertimento e partecipazione nelle attività quotidiane ricche di azioni monotone e noiose attraverso il gioco.}
}
\newglossaryentry{SDK}
{
    name=SDK,
    sort=sdk,
    description={Un Software Developer Kit è un insieme di strumenti per lo sviluppo e la documentazione di software}
}

\newglossaryentry{SRP}
{
    name=SRP,
    sort=srp,
    description={Il Single Responsability Princile afferma che ogni elemento di un programma deve avere una sola responsabilità, e che tale responsabilità debba essere interamente incapsulata dall'elemento stesso. Tutti i servizi offerti dall'elemento dovrebbero essere strettamente allineati a tale responsabilità. }
}
\newglossaryentry{codebase}
{
    name=codebase,
    sort=codebase,
    description={Collezione di codice sorgente usata per costruire una particolare applicazione o un particolare componente.}
}
 % database di termini
\makeglossaries


%**************************************************************
% Impostazioni di graphicx
%**************************************************************
\graphicspath{{immagini/}} % cartella dove sono riposte le immagini


%**************************************************************
% Impostazioni di hyperref
%**************************************************************
\hypersetup{
    %hyperfootnotes=false,
    %pdfpagelabels,
    %draft,	% = elimina tutti i link (utile per stampe in bianco e nero)
    colorlinks=true,
    linktocpage=true,
    pdfstartpage=1,
    pdfstartview=,
    % decommenta la riga seguente per avere link in nero (per esempio per la stampa in bianco e nero)
    %colorlinks=false, linktocpage=false, pdfborder={0 0 0}, pdfstartpage=1, pdfstartview=FitV,
    breaklinks=true,
    pdfpagemode=UseNone,
    pageanchor=true,
    pdfpagemode=UseOutlines,
    plainpages=false,
    bookmarksnumbered,
    bookmarksopen=true,
    bookmarksopenlevel=1,
    hypertexnames=true,
    pdfhighlight=/O,
    %nesting=true,
    %frenchlinks,
    urlcolor=webbrown,
    linkcolor=RoyalBlue,
    citecolor=webgreen,
    %pagecolor=RoyalBlue,
    %urlcolor=Black, linkcolor=Black, citecolor=Black, %pagecolor=Black,
    %pdftitle={\myTitle},
    %pdfauthor={\textcopyright\ \myName, \myUni, \myFaculty},
    %pdfsubject={},
    %pdfkeywords={},
    %pdfcreator={pdfLaTeX},
    %pdfproducer={LaTeX}
}

%**************************************************************
% Impostazioni di itemize
%**************************************************************
\renewcommand{\labelitemi}{$\bullet$}

%\renewcommand{\labelitemi}{$\bullet$}
%\renewcommand{\labelitemii}{$\cdot$}
%\renewcommand{\labelitemiii}{$\diamond$}
%\renewcommand{\labelitemiv}{$\ast$}


%**************************************************************
% Impostazioni di listings
%**************************************************************
\lstset{
    language=[LaTeX]Tex,%C++,
    keywordstyle=\color{RoyalBlue}, %\bfseries,
    basicstyle=\small\ttfamily,
    %identifierstyle=\color{NavyBlue},
    commentstyle=\color{Green}\ttfamily,
    stringstyle=\rmfamily,
    numbers=none, %left,%
    numberstyle=\scriptsize, %\tiny
    stepnumber=5,
    numbersep=8pt,
    showstringspaces=false,
    breaklines=true,
    frameround=ftff,
    frame=single
} 
\colorlet{punct}{red!60!black}
\definecolor{background}{HTML}{EEEEEE}
\definecolor{delim}{RGB}{20,105,176}
\colorlet{numb}{magenta!60!black}
\definecolor{codegreen}{rgb}{0,0.6,0}
\definecolor{codegray}{rgb}{0.5,0.5,0.5}
\definecolor{codepurple}{rgb}{0.82, 0.62, 0.91}
\definecolor{codeblue}{rgb}{0, 0.5, 1}
\definecolor{codeorange}{rgb}{1, 0.22, 0}

% Define the codebox
\lstdefinelanguage{dart}{
backgroundcolor=\color{background},   
    commentstyle=\color{codegreen},
    basicstyle=\ttfamily\footnotesize,
    breakatwhitespace=false,         
    breaklines=true, 
    stepnumber=1,
    captionpos=b,                    
    keepspaces=true,                 
    numbers=left,                    
    numbersep=8pt,                  
    showspaces=false,                
    showstringspaces=false,
    showtabs=false,                  
    tabsize=2,
    %frame=shadowbox
    emph=[1]{CheckLogin, HookWidget, Widget, BuildContext, bool, String, int, List, GenericErrorPage, LoginScreen, LoadUser, LoadCoach, DelayedWidget, Widget, Key, this, super, Future, false, true, Duration, Waiter, List, Goals, Appointment, ChangeNotifierProvider, UserRepo, Provider, ChangeNotifier, Reader, User, Options, Dio, Future },% Insert here the types you are using
    emphstyle=[1]{\color{codeblue}},
    emph=[2]{useProvider, build, useState, useEffect, delayed, then, buildCards, addAll, buildGoalsCards, where, map, toList, buildAppointmentCards, getAccessToken, fetchUser, watch, notifyListeners},% Insert here the methods you are using
    emphstyle=[2]{\color{codepurple}},
    emph=[3]{class, extends, @override, if, return, else, switch, case, default, final, required, is, as, =>, async, void, try, catch, finally, await},% Insert here the keywords you are using
    emphstyle=[3]{\color{codeorange}},%
}

\lstdefinelanguage{json}{
    basicstyle=\ttfamily\footnotesize,
    numbers=left,
    numberstyle=\scriptsize,
    stepnumber=1,
    numbersep=8pt,
    showstringspaces=false,
    breaklines=true,
    backgroundcolor=\color{background},
    string=[s]{"}{"},
    stringstyle=\color{blue},
    comment=[l]{:},
    commentstyle=\color{black},
    literate=
     *{0}{{{\color{numb}0}}}{1}
      {1}{{{\color{numb}1}}}{1}
      {2}{{{\color{numb}2}}}{1}
      {3}{{{\color{numb}3}}}{1}
      {4}{{{\color{numb}4}}}{1}
      {5}{{{\color{numb}5}}}{1}
      {6}{{{\color{numb}6}}}{1}
      {7}{{{\color{numb}7}}}{1}
      {8}{{{\color{numb}8}}}{1}
      {9}{{{\color{numb}9}}}{1}
      {:}{{{\color{punct}{:}}}}{1}
      {,}{{{\color{punct}{,}}}}{1}
      {[}{{{\color{delim}{[}}}}{1}
      {]}{{{\color{delim}{]}}}}{1},
}


%**************************************************************
% Impostazioni di xcolor
%**************************************************************
\definecolor{webgreen}{rgb}{0,.5,0}
\definecolor{webbrown}{rgb}{.6,0,0}


%**************************************************************
% Altro
%**************************************************************

\newcommand{\omissis}{[\dots\negthinspace]} % produce [...]

% eccezioni all'algoritmo di sillabazione
\hyphenation
{
    ma-cro-istru-zio-ne
    gi-ral-din
}

\newcommand{\sectionname}{sezione}
\addto\captionsitalian{\renewcommand{\figurename}{Figura}
                       \renewcommand{\tablename}{Tabella}}

\newcommand{\glsfirstoccur}{\ap{{[g]}}}

\newcommand{\intro}[1]{\emph{\textsf{#1}}}

%**************************************************************
% Environment per ``rischi''
%**************************************************************
\newcounter{riskcounter}                % define a counter
\setcounter{riskcounter}{0}             % set the counter to some initial value

%%%% Parameters
% #1: Title
\newenvironment{risk}[1]{
    \refstepcounter{riskcounter}        % increment counter
    \par \noindent                      % start new paragraph
    \textbf{\arabic{riskcounter}. #1}   % display the title before the 
                                        % content of the environment is displayed 
}{
    \par\medskip
}

\newcommand{\riskname}{Rischio}

\newcommand{\riskdescription}[1]{\textbf{\\Descrizione:} #1.}

\newcommand{\risksolution}[1]{\textbf{\\Soluzione:} #1.}

%**************************************************************
% Environment per ``use case''
%**************************************************************
\newcounter{usecasecounter}             % define a counter
\setcounter{usecasecounter}{0}          % set the counter to some initial value

%%%% Parameters
% #1: ID
% #2: Nome
\newenvironment{usecase}[2]{
    \renewcommand{\theusecasecounter}{\usecasename #1}  % this is where the display of 
                                                        % the counter is overwritten/modified
    \refstepcounter{usecasecounter}             % increment counter
    \vspace{10pt}
    \par \noindent                              % start new paragraph
    {\large \textbf{\usecasename #1: #2}}       % display the title before the 
                                                % content of the environment is displayed 
    \medskip
}{
    \medskip
}

\newcommand{\usecasename}{UC}

\newcommand{\usecaseactors}[1]{\textbf{\\Attori Principali:} #1. \vspace{4pt}}
\newcommand{\usecasepre}[1]{\textbf{\\Precondizioni:} #1. \vspace{4pt}}
\newcommand{\usecasedesc}[1]{\textbf{\\Descrizione:} #1. \vspace{4pt}}
\newcommand{\usecasepost}[1]{\textbf{\\Postcondizioni:} #1. \vspace{4pt}}
\newcommand{\usecasealt}[1]{\textbf{\\Scenario Alternativo:} #1. \vspace{4pt}}

%**************************************************************
% Environment per ``namespace description''
%**************************************************************

\newenvironment{namespacedesc}{
    \vspace{10pt}
    \par \noindent                              % start new paragraph
    \begin{description} 
}{
    \end{description}
    \medskip
}

\newcommand{\classdesc}[2]{\item[\textbf{#1:}] #2}
%**************************************************************
%cose mie
\definecolor{red}{cmyk}{0,0.87,0.68,0.32}
\newcolumntype{C}[1]{>{\centering\let\newline\\\arraybackslash\hspace{0pt}}m{#1}}
