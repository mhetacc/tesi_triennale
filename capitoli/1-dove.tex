% !TEX encoding = UTF-8
% !TEX TS-program = pdflatex
% !TEX root = ../tesi.tex

%**************************************************************
\chapter{Contesto Aziendale}
\label{cap:contesto-aziendale}
%**************************************************************

\section{L'azienda}
\subsection{Descrizione}
Le informazioni a seguito sono state prese direttamente dall'azienda, tuttavia ritengono siano rappresentative della realtà considerando sia il progetto al quale ho partecipato, sia il resto dell'offerta \textit{software}.\\
\textbf{Datasoil S.r.l.} è una \textit{startup} innovativa che si occupa di sviluppare applicativi dedicati alla gestione aziendale, altamente integrati con industria 4.0 e \textit{digital twin}: vengono impiegati \textit{machine learning} e analisi predittive per ottenere analitiche migliori rispetto alla concorrenza, con un occhio attento a innovazione, sicurezza e scalabilità, ottenute tramite prototipazione e \textit{Infrastrucutre as Code}, delegando quindi la gestione delle componenti \textit{server} a servizi terzi come \aws{} (ecosistema \textit{software} fornito da Amazon, che va da semplici piattaforme di \textit{cloud storage} fino a sistemi predittivi basati su \textit{machine learning}).\\
Il loro obiettivo è manipolare ed elaborare dati di natura inerentemente caotica, ordinarli e presentarli all'utente finale tramite interfacce grafiche intuitive, promuovendo un'interazione efficace tra utilizzatore e mezzo. Prediligono un approccio ai problemi condiviso e modulare per far nascere soluzioni inedite da applicare a grandi progetti, forti di un \textit{know-how} che abbraccia esigenze verticali (vengono impiegate internamente molte tecnologie sia lato \textit{frontend} che lato \textit{backend}).\\
Le piattaforme dedicate ad \textit{industry} 4.0, \textit{smart building} e \textit{smart city} poggiano su dei \textit{data silos} attualmente esistenti, garantendo un'interpretazione integrata e di alto livello delle informazioni aziendali fondendole con eventi provenienti dai diversi \textit{business level} creando \textit{insight} proattivi in tempo reale. I sistemi predittivi di Datasoil sono in grado di notificare la persona giusta al momento giusto anticipando la gestione delle criticità interfunzionali.\\
Ho avuto modo di verificare molte di queste affermazioni direttamente tramite il progetto proposto, in quanto si tratta di un applicativo legato all'industria 4.0 che mappa \textit{asset} aziendali (come ad esempio macchinari) ad un gemello digitale, permettendo sia di localizzarli su una mappa tridimensionale sia, obiettivo dello stage, di visualizzarli in realtà aumentata.\\

\subsection{Clientela}
Datasoil sviluppa applicazioni che guardano al mondo professionale più che a quello generico \textit{consumer}, ad esempio in questo specifico caso l'applicativo (che prende il nome di MobileSYN) è dedicato a un'utenza strettamente aziendale, e ci si aspetta che venga utilizzata da operatori interni con dispositivi controllati (generalmente \textit{tablet} Apple). \\
Anche un altro prodotto dell'azienda, chiamato LifestyleSync, mira a fornire uno strumento per migliorare il benessere dei dipendenti nelle aziende, permettendo a ogni dipendente di essere seguito da un \textit{coach} che, tramite un'applicazione di supporto LSCoach, può progettare allenamenti personalizzati e vedere la progressione degli stessi.

\subsection{Processi Interni}

\subsection{Innovazione}
%************************************************************************************************************
\section{Tecnologie e Strumenti}
\label{sec:tecnologie-e-strumenti}

\subsection{Tecnologie e Strumenti di sviluppo}
\subsubsection{Visual Studio Code}
\vsc{} è un editor per codice sorgente, sviluppato da \textit{Microsoft} servendosi del framework \textit{Electron}, disponibile per Windows, Linux e macOS. Le sue funzionalità comprendo supporto per debugging, syntax highlighting, intelligent code completion, snippets, code refactoring, e integrazione con Git. \\
Gli utenti possono configurare tema, macro, preferenze e installare estensioni che ne aumentano le funzionalità aggiungendo ad esempio il supporto ai maggiori linguaggi di porgrammazione attualmente presenti.\\
Nella \textit{Stack Overflow 2021 Developer Survey}, \vsc{} è risultato essere l'IDE più popolare, adottato dal 70\% degli intervistati. {\tiny Fonte: \href{https://en.wikipedia.org/wiki/Visual_Studio_Code}{Wikipedia}}\\
\E{} inoltre presente e molto comoda l'estensione di \flutter{} che, tra le altre cose, permette di gestire direttamente in gui i vari dispositivi (smartphone, browser o emulatore) sui quali installare e lanciare l'app che si sta programmando.

\subsubsection{Android Studio}
\textit{Android Studio} è l'IDE ufficiale del sistema operativo di Gooogle, sostituendo Eclipse dal 2015, costruito sul software IntelliJ di JetBrains e progettato specificatamente per lo sviluppo Android. \E{} disponibile per Windows, Linux e macOS.\\
Il 7 maggio 2019 Kotlin rimpiazzò Java come linguaggio consigliato per Android, favorendo un'integrazione ancora maggiore con JetBrains che, appunto, ha sviluppato e prodotto Kotlin stesso. {\tiny Fonte: \href{https://en.wikipedia.org/wiki/Android_Studio}{Wikipedia}}\\
Personalmente ho cercato nonostante tutto di programmare il più possibile su \vsc{} preferendolo quindi ad \textit{Android Studio} in virtù della sua maggiore leggerezza e delle sue più ampie possibilità di personalizzazione.

\subsubsection{Dart}
Dart è un linguaggio di programmazione sviluppato da Google per il client development, ovvero per web e mobile app, e può anche essere impiegato per la costruzione di applicazione desktop e server.\\
\E{} un linguaggio object-oriented, class-based, garbage-collected, strongly-typed e con una sintassi nello stile di C. Può compilare sia codice macchina che JavaScript e supporta interefacce, mixins, classi astratte, refined generics e type inference.{\tiny Fonte: \href{https://en.wikipedia.org/wiki/Dart_(programming_language)}{Wikipedia}}\\
Personalmente l'ho trovato elegante e conciso nella sua struttura, e presenta dei vantaggi consistenti come la null protection e le espressioni ternarie.

\subsubsection{Flutter}
Flutter è un framework open-source creato da Google per lo sviluppo della UI di applicazioni cross-platform per Android, iOS, Linux, macOS, Windows, Google Fuchsia e il web partendo da un singolo codebase.\\
Rialsciato a maggio 2017, le applicazioni in Flutter sono scritte in Dart e sfruttano molte delle features avanzate del linguaggio, e le componenti principali del framework sono:
\begin{itemize}
    \item \textbf{Dart platform:} Flutter gira all'interno della Dart Virtual Machine, che si serve di un execution engine just-in-time;
    \item \textbf{Flutter engine:} scritto primariamente in C++, fornisce supporto a basso livello per il rendering e implementa accessibilità, file e network I/O, supporto nativo per i plugin e molto altro;
    \item \textbf{Foundation library:} scritta in Dart, fornisce classi e funzioni di base che vengono usate per costruire gli applcativi in Flutter, come ad esempio le API per la comunicazione con l'engine;
    \item \textbf{Design-specific widgets:} Flutter contiene due famiglie di widget che si conformano a delle specifiche scuole di design: \textit{Material} per Google e \textit{Cupertino} per Apple;
    \item \textbf{Flutter DevTools:} famiglia di tools generici che vengono sfruttati per lo sviluppo di software in Flutter.
\end{itemize}
~\\
Inoltre, una libreria fondamentale per semplificare l'utilizzo di Flutter è \textit{Flutter Hooks}: lo scopo è implementare delle strutture simili agli Hooks di React, consentono di sostituire gli Stateful Widget riducendo il boilerplate code e semplificando la gestione di side-effect e ciclo di vita dei widget grazie alla funzione \textit{useEffect}.

\subsubsection{Java}
Java è un linguaggio di programmazione ad alto livello object-oriented progettato per avere il minor numero possibile di dipendenze implementative (favorendo quindi l'uso di interfacce e classi astratte) e, soprattutto, per aderire al motto \textit{"write once, run anywhere"}: il bytecode di Java può infatti girare senza necessità di ricompilazione su qualsiasi piattaforma che supporti la Java Virtual Machine (comunemente abbreviata in JVM).\\  
La sintassi è simile a quelle di C o C++, ma fornisce funzionalità dinamiche generalmente inedite per i linguaggi compilati, come reflection e modifica del codice a runtime.\\
Originalmente rilasciato nel 1995 come componente centrale della piattaforma Java per Sun Microsystems e sotto licenza proprietaria, dal 2007 la maggior parte dei suoi componenti sono diventati open-source sotto l'ombrello della \textit{GPL-2.0-only} (licenza specifica per software libero pubblicata dalla Free Software Foundation).\\
Sebbene Oracle (attuale proprietario di Sun Microsystems) offra un'implementazione proprietaria chiamata \textit{HotSpot JVM}, la specifica ufficiale è la \textit{OpenJDK JVM} che è gratuita, open-source e di default nella maggior parte delle distribuzioni Linux.

\subsubsection{Kotlin}
Kotlin è un linguaggio di programmazione cross-platform, staticamente tipizzato, progettato per avere completa interoperabilità con Java, gode di una sintassi più concisa di quest'ultimo grazie alla type inference di cui è dotato e dipende dalla \textit{Java Class Library} per la versione della JVM da usare.\\
\E{} inoltre in grado di compilare codice in JavaScript (sfruttato poi da applicazioni frontend) o codice nativo tramite \textit{LLVM} (per app iOS che condividono la business logic con applicazioni Android).\\
Incluso in Android Studio come alternativa al compilatore standard Java dal 2017, produce bytecode Java 8 di default ma permette al programmatore di scegliere delle versioni target successive (fino a Java 18).

\subsubsection{Azure Spatial Anchors}
Azure Spatial Anchors è un servizio di Microsoft mirato a fornire strumenti per sviluppare applicazioni in realtà aumentata su dispositivi iOS o Android predisposti per, rispettivamente, ARKit e ARCore, oppure su Microsoft HoloLens.\\
Permette di riconoscere spazi tridimensionali all'interno dei quali posizionare punti di interesse, chiamati Spatial Anchors, che possono essere salvati in cloud e poi recuperati, e possono essere messi in relazione a oggetti reali (come macchinari in un contesto industriale) o virtuali.

\subsubsection{ARCore}
ARCore, conosciuto anche come Google Play Services per AR, è un software development kit prodotto da Google per permettere lo sviluppo di applicazioni in realtà aumentata.\\
Si serve di tre componenti principali:
\begin{itemize}
    \item \textbf{6DOF:} il dispositivo sfrutta una piattaforma inerziale a sei assi per comprendere e tracciare la propria posizione relativa allo spazio;
    \item \textbf{Environmental understanding:} permette riconoscere dimensioni e locazione delle superfici lisce;
    \item \textbf{Light estimation:} permette al telefono di stimare le condizioni di luce dell'ambiente.
\end{itemize}
asa si appoggia proprio ad ARCore per implementare le proprie funzionalità di AR su Android.

\todo
\textcolor{todoOrange}{ar flutter plugin, sceneform?}


%**************************************************************
\subsection{Strumenti organizzativi}

\subsubsection{Slack}
Slack è una piattaforma di comunicazione istantanea di proprietà di Salesforce e sviluppata da Slack Technologies per Windows, Linux, macOS, Android e iOS.\\
Permette di comunicare tramite messaggi, chiamate vocali e video, e di inviare media e files nelle chat private o nei canali. Quest'ultimi fungono da aggregatori e permettono di suddividere tematicamente la comunicazione.\\
\E{} inoltre possibile suddividere a un livello ancora più alto tramite i \textit{workspaces}, che aggregano al proprio interno utenti, canali e applicazioni: sono proprio quest'ultime che mostrano con maggiore chiarezza l'orientamento \textit{corporate} di Slack, infatti fornisce integrazione nativa con i maggiori software gestionali e organizzativi (come ad esempio Jira, Zoom, Drive, Outlook, etc).

\subsubsection{Github}
GitHub è il più grande servizio di hosting di codice sorgente al mondo, abitualmente usato per sviluppare progetti open source, e fornisce vari servizi: version control di Git distribuito e access control, bug tracking, issue tracking, richieste di modifiche software, task management, integrazione continua e wiki per ogni progetto.\\ 
Permette inoltre, tramite forking, pull request e commenti, il miglioramento collaborativo del software, ad esempio risolvendo bug, aggiungendo funzionalità o migliorando quelle presenti.

