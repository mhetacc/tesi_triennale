% !TEX encoding = UTF-8
% !TEX TS-program = pdflatex
% !TEX root = ../tesi.tex

%**************************************************************
\chapter{Progettazione}
\label{cap:progettazione}
%**************************************************************
\section{Organizzazione repository e sistema di ticketing}
\label{sec:organizzazione-repository-sistema-ticketing}

A livello di organizzazione del repository si è deciso di adottare lo stile generalmente utilizzato dall'azienda.\\
Partendo da una base stabile dell'applicazione si procede in modo incrementale per aggiunta di funzionalità seguendo un approccio \gls{agile}.\\
Ogni nuova funzionalità o \textit{issue} viene segnalata attraverso Jira da un membro del team, il quale si occupa anche di fornire un titolo, una breve descrizione e di assegnarla ad uno sviluppatore frontend o backend, in base al tipo di funzionalità. Jira automaticamente crea un ticket con un codice identificativo e la descrizione del problema.\\
Una volta preso in carico un ticket, a livello di repository, si procede creando un nuovo \gls{branch} con nome uguale al codice identificativo della \textit{issue}. Al termine dello sviluppo per consolidare il lavoro svolto nel \gls{branch} viene creata una \gls{pull request} verso il \gls{branch} principale, ossia il \textit{main}. La richiesta, una volta controllata e priva di conflitti, viene confermata e tutto il codice prodotto viene unito nel \gls{branch} principale, così da avere sempre un prodotto stabile e aggiornato all'ultima funzionalità aggiunta.

\begin{figure}[ht]
    \centering
    \includegraphics[height=6.5cm]{ticket}
    \caption{Esempio di ticket su Jira}
\end{figure}

Si è deciso inoltre di mantenere entrambe le applicazioni sullo stesso \textit{repository}, per questo motivo si sono dovute organizzare le cartelle in modo da avere un ambiente di lavoro ordinato. La scelta è stata quella di mantenere nella \textit{root} del progetto delle cartelle contenti i file condivisi tra le due applicazioni, e di creare altre due cartelle: \textit{client} e \textit{coach} aventi come figli delle cartelle con lo stesso nome di quelle presenti nella \textit{root}. In queste ultime cartelle vengono raccolti i file utilizzati solamente da una delle due applicazioni.
La struttura è risultata quindi la seguente:
\begin{itemize}
    \item auth, contiene il Provider per l'autenticazione di entrambe le applicazioni
    \item models, contiene le classi del modello di entrambe le applicazioni
    \item pages, contiene i widget \textit{fullscreen} utilizzati da entrambe le applicazioni
    \item widget, contiene i widget utilizzati da entrambe le applicazioni
    \item client, cartella contenente i file per l'applicazione client
    \begin{itemize}
        \item providers, contiene i provider dell'app client
        \item models, contiene le classi del modello dell'app client
        \item pages, contiene i widget \textit{fullscreen} dell'app client
        \item widget, contiene i widget utilizzati dell'app client
    \end{itemize}
    \item coach, cartella contenente i file per l'applicazione coach
    \begin{itemize}
        \item providers, contiene i provider dell'app coach
        \item models, contiene le classi del modello dell'app coach
        \item pages, contiene i widget \textit{fullscreen} dell'app coach
        \item widget, contiene i widget utilizzati dell'app coach
    \end{itemize}
\end{itemize}
%**************************************************************
\section{Scelta framework Flutter}
\label{sec:scelta-framework-flutter}
Inizialmente l'azienda non aveva ancora deciso se l'app sarebbe stata sviluppata con \textit{Flutter} o con \textit{React Native}, framework che loro utilizzano quotidianamente per altri progetti.\\ 
Sono state quindi analizzate le principali differenze tra i due framework fino a giungere alla conclusione di utilizzare Flutter: sia perché ritenuto su carta il più performante, sia per testarne i limiti.\\
Essendo infatti questo un progetto di dimensioni ridotte si è deciso di provare Flutter per testarne le potenzialità e capire se possa essere utilizzato dall'azienda anche per applicazioni di dimensioni maggiori.
%----------------------------------------------------------------
\section{Onboarding e primo login}
\label{sec:progettazione-onboarding-primo-login}
Come prima parte dell'applicazione ho concluso la progettazione delle funzionalità onboarding e di primo login dell'utente, che in parte era già stata effettuata dal mio tutor.
%----------------------------------------------------------------
\subsection{Autenticazione tramite auth0}
\label{sec:autenticazione-tramite-auth0}
Per l'autenticazione si è deciso di appoggiarsi al servizio Auth0, già ampiamente utilizzato dall'azienda.\\
Grazie ad Auth0 ci è possibile delegare ad un servizio terzo la gestione dell'autenticazione attraverso diversi servizi come Google, Facebook o semplicemente con nome utente e password.\\
Auth0 fornisce una pagina web di autenticazione unica per qualsiasi provider indicato. A login avvenuto, l’applicazione riceve dall’API di Auth0 un ID token ed un Refresh Token.\\ Mediante l’ID token sarà possibile accedere alle API LifestyleSync tramite autenticazione \gls{JWT}, includendolo in ogni header delle chiamate HTTP. In questo modo il backend può verificare che il token sia presente nel registro di Auth0 e che l’utente abbia i permessi per accedere ai dati. Tuttavia, prima di effettuare qualsiasi richiesta al backend, è buona pratica controllare la scadenza temporale indicata in ID token: se questo è scaduto è possibile utilizzare Refresh Token per chiedere ad Auth0 un nuovo ID token. Rinnovando i token prima della scadenza è quindi possibile mantenere l’accesso all’applicazione inserendo le credenziali soltanto al primo login.

\begin{figure}[h!]
    \centering
    \includegraphics[height=10cm]{auth0}
    \caption{Diagramma di sequenza dell autenticazione usando auth0}
\end{figure}
\subsection{Chiamate asincrone}
Un punto chiave per lo sviluppo dell'app è quello delle chiamate asincrone.\\
Quando una funzione richiede molto tempo per essere completata, come ad esempio una funzione che richiede dati ad un un server, è opportuno chiamarla in modo asincrono, così da non fermare tutta l'applicazione in attesa del completamento di quella funzione. Utilizzando le chiamate asincrone infatti il flusso delle operazioni che non necessitano il termine della chiamata può continuare indisturbato.\\
Dart mette a disposizione due modi per la gestione di questo tipo di chiamate, noi in accordo comune abbiamo deciso di usare il metodo \textit{async} e \textit{await}: le funzioni che prevedono la possibilità di fare chiamate asincrone sono contrassegnate con la parola chiave \textit{async}. All'interno delle funzioni asincrone è possibile quindi usare la parola chiave \textit{await} che serve per dire al programma di attendere quell'istruzione prima di proseguire con la successiva. Se non viene utilizzato il comando \textit{await} allora tale funzione asincrona verrà eseguita parallelamente al proseguo del programma fino al suo completamento.\\
\begin{figure}[h!]
    \centering
    \includegraphics[height=6cm]{async}
    \caption{Chiamate sincrone VS asincrone}
\end{figure}
%*********************************************************************
\subsection{Richieste API}
Essendo l'applicazione un supporto ad una piattaforma di analisi dati la presenza di chiamate asincrone al backend è molto frequente, è stato quindi fondamentale progettare un pattern solido da riutilizzare per ogni chiamata.
Il \textit{pattern} da noi adottato prevede che una qualsiasi chiamata API CRUD possa trovarsi in tre diversi stati:
\begin{itemize}
    \item \textbf{Loading}: la chiamata è iniziata e non è ancora stata risolta;
    \item \textbf{Error}: la chiamata è terminata con un errore;
    \item \textbf{Response}: la chiamata è terminata e possiedo i dati di mio interesse.
\end{itemize}
Appena prima di inviare la richiesta HTTP l'applicazione viene messa in uno stato di \textit{loading}, nel caso di errori al backend o nessuna risposta ci si troverà in uno stato di \textit{error}, altrimenti, nel caso in cui il backend risponda con i dati corretti ci si troverà in uno stato di \textit{response}.\\
L'interfaccia dunque cambierà in base allo stato della chiamata, mostrando un indicatore di progresso mentre ci si trova in uno stato di \textit{loading}, mostrando un errore nel caso ci si trovasse in uno stato di \textit{error}, o mostrando i dati elaborati nel caso di \textit{response}.
\begin{figure}[h!]
    \centering
    \includegraphics[height=8cm]{api}
    \caption{Diagramma di sequenza di una richiesta API}
\end{figure}
\begin{lstlisting}[language=dart, firstnumber=1,caption={User provider}]
final userChangeNotifier =
    ChangeNotifierProvider<UserRepo>((ref) => UserRepo(ref.read));
///Fornisce lo stato di errore di [UserRepo].
final userError = Provider((ref) => ref.watch(userChangeNotifier).error);
///Fornisce il messaggio di errore di [UserRepo].
final userErrorMsg = Provider((ref) => ref.watch(userChangeNotifier).errorMsg);
///Fornisce lo stato di loading di [UserRepo].
final userIsLoading =
    Provider((ref) => ref.watch(userChangeNotifier).isLoading);
///Fornisce l'istanza di [User].
final userProvider = Provider((ref) => ref.watch(userChangeNotifier).user);

class UserRepo extends ChangeNotifier {
  bool error = false;
  bool isLoading = false;
  String errorMsg = "";
  final String apiUrl = dotenv.env['API_URL'] ?? "";
  User? user;
  final Reader read;
  UserRepo(this.read);
  Future<void> fetchUser() async {
    var token = await read(authProvider).getAccessToken();
    this.isLoading=true;
	notifyListeners();
    try {
      final response = await Dio().get(
        apiUrl + "/api/myself",
        options: Options(
            headers: 
                <String, String>{'Authorization': 'Bearer $token'}),);
      if (response.statusCode == 200) {
        this.user = User.fromJson(response.data));
        this.isLoading = false;
      } else {
        this.isLoading = false;
        this.error = value;
        this.errorMsg = "richiesta fallita";
      }
    } catch (e) {
      this.isLoading = false;
	  this.error = value;
      this.errorMsg = "richiesta fallita";
    } finally {
	  notifyListeners();
    }
  }
}
\end{lstlisting}
%*************************************************************
\subsection{API di LifestyleSync}
Di seguito sono riportate alcune delle API utilizzate durante lo sviluppo delle applicazioni.\\
Esse si rifanno ad un modello REST-like e non REST puro in quanto tutte le richieste HTTP vengono firmate mediante la tecnica \gls{JWT} presentata in precedenza.\\
Le API verranno riportate nella forma [metodo\textunderscore HTTP][url/api]
\begin{enumerate}
    \item \textbf{GET api/myself}\\
    Restituisce un file JSON che descrive l'utente, con le sue info e i dettagli del suo coach, con il seguente schema:
    \begin{lstlisting}[language=json,firstnumber=1,caption={JSON schema dell'API GET api/myself},captionpos=b]
    {
    "user_id": String,
    "given_name": String,
    "family_name": String,
    "email": String,
    "picture": String,
    "info":
        {
        survey_done: bool,
        coach: Map<String, dynamic>,
        oms_thresh: Map<String, int>,
        fit_profiles: Map<String, dynamic>
        }
    "user_metadata": Map<String, dynamic>
    }
    \end{lstlisting}
    dove: \begin{itemize}
        \item user\textunderscore id: è l'identificativo dell'utente nel database;
        \item given\textunderscore name: è il nome dell'utente;
        \item family\textunderscore name: è il cognome dell'utente;
        \item email: è la e-mail dell'utente;
        \item picture: è il link dell'immagine del profilo dell'utente;
        \item user\textunderscore metadata: sono i meta-dati dell'utente;
        \item info:
        \begin{itemize}
            \item survey\textunderscore done: è true se l'utente ha completato il questionario;
            \item coach: sono i dati del coach dell'utente;
            \item oms\textunderscore thresh: sono le soglie da raggiungere secondo l'OMS;
            \item fit\textunderscore profiles: sono i profili fit dell'utente;
        \end{itemize}
    \end{itemize}
    \item \textbf{GET api/goals}\\
    Questa API accetta diversi parametri obbligatori:
    \begin{itemize}
        \item end: data di fine dei goals che voglio in formato \textit{IsoTime};
        \item active: se voglio i goals attivi o meno in formato \textit{Bool};
        \item interval: intervallo di tempo antecedente a \textit{end}, può essere: week, month, 3month o 6month in formato \textit{String}.
    \end{itemize}
    Restituisce un array di oggetti JSON che descrivono i goals dell'utente, con il seguente schema:
    \begin{lstlisting}[language=json,firstnumber=1,caption={JSON schema dell'API GET api/goals},captionpos=b]
    {
    "id": String,
    "u_id": String,
    "goals":
        {
        "calories":
            {
            "m": String,
            "thresh": int,
            }    
        "duration_activities":
            {
            "m": String,
            "thresh": int,
            }    
        "duration_normal":
            {
            "m": String,
            "thresh": int,
            }    
        "steps":
            {
            "m": String,
            "thresh": int,
            }    
        }
    "status":
        {
        "calories": int,
        "duration_activities": int,
        "duration_normal": int,
        "steps": int,
        }
    "start": IsoTime,
    "end": IsoTime,
    "achieved": bool,
    "active": bool,
    }
    \end{lstlisting}
    dove: \begin{itemize}
        \item id: è l'identificativo dell'obiettivo nel database;
        \item u\textunderscore id: è l'identificativo dell'utente nel database;
        \item goals:
        \begin{itemize}
            \item calories: sono le calorie da bruciare, \textit{m} è il tipo di metrica e \textit{thresh} è la soglia;
            \item duration\textunderscore activities: sono i minuti di attività da raggiungere, \textit{m} è il tipo di metrica e \textit{thresh} è la soglia;
            \item duration\textunderscore normal: sono i minuti attivi normali da raggiungere, \textit{m} è il tipo di metrica e \textit{thresh} è la soglia;
            \item steps: sono i passi da fare, \textit{m} è il tipo di metrica e \textit{thresh} è la soglia;
        \end{itemize}
        \item status:
        \begin{itemize}
            \item calories: sono le calorie bruciate;
            \item duration\textunderscore activities: sono i minuti di attività raggiunti;
            \item duration\textunderscore normal: sono i minuti attivi normali raggiunti;
            \item steps: sono i passi fatti;
        \end{itemize}
        \item start: è la data di inizio dell'obiettivo;
        \item end: è la data di fine dell'obiettivo;
        \item achieved: indica se l'obiettivo è stato raggiunto;
        \item active: indica se l'obiettivo è attivo.
    \end{itemize}
    \item \textbf{GET api/agendas}\\
    Questa API accetta diversi parametri obbligatori:
    \begin{itemize}
        \item end: data di fine degli appuntamenti che voglio in formato \textit{IsoTime};
        \item interval: intervallo di tempo antecedente a \textit{end}, può essere: week, month, 3month o 6month in formato \textit{String}.
    \end{itemize}
    Restituisce un array di oggetti JSON che descrivono gli appuntamenti, con il seguente schema:
    \begin{lstlisting}[language=json,firstnumber=1,caption={JSON schema dell'API GET api/agendas},captionpos=b]
    {
    "id": String,
    "start": IsoTime,
    "end": IsoTime,
    "subject": String,
    "tid": String,
    "notes": String,
    "coach": Map<String, dynamic>,
    "client": Map<String, dynamic>,
    }
    \end{lstlisting}
    dove: \begin{itemize}
        \item id: è l'identificativo dell'appuntamento nel database;
        \item start: è la data e ora di inizio dell'appuntamento;
        \item end: è la data e ora di fine dell'appuntamento;
        \item subject: è il titolo dell'appuntamento;
        \item notes: sono le note dell'appuntamento;
        \item tid: è il tenant a cui appartiene l'appuntamento;
        \item coach: sono i dettagli del coach;
        \item client: sono i dettagli del cliente.
    \end{itemize}
\end{enumerate}
%************************************************************
\subsection{Stato globale e locale}
Una delle problematiche principali quando si deve sviluppare un applicazione mobile o web è il come salvare i dati e gestire lo stato dell'applicazione.\\

Flutter prevede l’esistenza di due tipi di stato (simili ad uno \textit{store}): lo stato globale dell’applicazione, che ad ogni update causa un update dell’intero albero di \textit{rendering} dell’applicazione e lo stato locale, proprio del singolo widget e il cui update causa l’aggiornamento unicamente del widget stesso e dei suoi eventuali figli.\\
A seguito dell’analisi dei requisiti e del comportamento dell’applicazione si è deciso di utilizzare lo stato globale per i dati utilizzati da più widget con o senza legami di parentela. Mentre, seguendo il pattern \gls{SRP}, lo stato locale viene utilizzato dai widget che non condividono dati con altri componenti (ad esempio widget di pura presentazione o che rappresentano dati provenienti da un’API non sfruttata dall’intero applicativo).\\

Per la gestione dello stato globale abbiamo usato il provider pattern. Questo prevede l'utilizzo di un oggetto contenente i dati dell'applicazione che notifica ogni cambio di stato, il quale viene fornito all'applicazione utilizzando un \textit{provider}. In questo modo l'interfaccia grafica sarà sempre aggiornata con il valore corrente dello stato.\\
Per la gestione dello store globale si è deciso di utilizzare la libreria \textit{Riverpod}, questa fornisce una vasta scelta di \textit{provider} diversi, e rende più facile l'accesso ai dati grazie ad un hook da loro predisposto: \textit{useProvider}.
Per lo store locale invece, in seguito alla scelta di usare Flutter Hooks, abbiamo sfruttato l'hook \textit{useState}, questo costruisce una variabile che viene osservata dal componente, ad ogni cambiamento di questa variabile il componente si ricostruisce, aggiornandosi con il valore corrente dello stato.
%**************************************************************
\subsection{Routing iniziale}
Per questa prima parte del progetto è stato fondamentale progettare un buon routing per gestire correttamente i diversi casi di accesso.\\
Si è deciso per provare automaticamente il login all'avvio dell'applicazione utilizzando un token salvato nel \textit{secure storage} dello smartphone. Nel caso di accesso avvenuto con successo si prosegue, altrimenti si rimanda l'utente ad una schermata di login.\\
A questo punto è necessario controllare se è la prima volta che il cliente accede all'applicazione o meno, per fare questo controllo vengono richieste al \textbf{backend} le informazioni dell'utente e nel mentre viene mostrata una schermata di caricamento, a questo punto le possibili alternative di routing sono:
\begin{itemize}
    \item \textbf{Primo accesso}: viene mostrata all'utente una breve introduzione all'applicazione, gli viene presentato il suo coach, gli viene fatto rispondere ad un questionario rispetto al suo stile di vita e infine viene mandato all'homepage. Nel caso in cui l'esecuzione dell'app venisse interrotta prima del termine del questionario, l'accesso successivo sarà considerato nuovamente un primo accesso.
    \item \textbf{Altro accesso}: recupero i dati del cliente a cui viene mostrata l'homepage. In questo modo l'utente non ha più la possibilità di vedere l'introduzione all'app e di rispondere al questionario.
\end{itemize}
%***************************************************************
\section{Connessione ai device}
Altra tematica principale durante la fase di progettazione è stata quella della connessione della nostra app ai profili dei provider di fitness band e altri dispositivi in grado di rilevare passi, frequenza cardiaca e attività fisiche.\\
L’applicazione non prevede nessuna connessione diretta ai dispositivi, bensì ai profili dell’utente con i quali accede alle funzionalità di raccolta dati offerte dai provider. Ad esempio per rilevare i dati di durata di attività fisica e distanza percorsa da un utente che possiede una smartband Garmin, l’applicazione offrirà la possibilità di connettere il profilo LifestyleSync a quello Garmin.\\
Questa tematica era già stata trattata dall'azienda prima del mio arrivo e la soluzione scelta è stata quella di connettere la nostra app non direttamente ai device, ma all'account nelle rispettive app dei diversi device. Ossia, per rilevare i dati di una smartband di Garmin andiamo a collegare l'account del client in LifestyleSync al suo account nell'app Garmin.\\
I servizi che attualmente si prevede di utilizzare sono: Google Fit, Fitbit, e Garmin.\\
L'idea è stata quindi quella di progettare una sezione all'interno dell'app, in cui l'utente possa vedere a quali account è già associato e potrà quindi disconnettersi, e a quali account può collegarsi. Cliccando su uno di questi nel caso di disconnessione verrà fatta una richiesta POST al \textit{backend} che si occuperà della disconnessione, mentre nel caso di connessione l'utente verrà reindirizzato alla pagina di login del servizio scelto, una volta fatto il login verrà riportato alla pagina dell'app LifestyleSync e tramite una richiesta POST al \textit{backend} verrà fatta l'associazione.
%**************************************************************
\section{Dashboard utente}
L'app dovrà avere diverse sezioni principali: \textit{Homepage}, \textit{Move}, \textit{Food}, \textit{Health}; e altre secondarie: \textit{Agenda}, \textit{Chat}.\\
Abbiamo quindi fin da subito pensato di sfruttare una struttura a pagine.\\
Per fare ciò si è resa quindi necessaria una barra di navigazione e un componente per scorrere tra le diverse pagine.\\
Da qui si sono progettate ad una ad una tutte le diverse schermate.
\subsection{Homepage}
È la pagina principale, qui l'utente deve poter vedere un breve riassunto del suo percorso e del suo stato di movimento.\\
Dovrà quindi essere presente il nome del cliente con la sua immagine del profilo e un piccolo badge con un resoconto degli obiettivi in corso, falliti, completati e il badge relativo all'ultimo \textit{achievement} sbloccato.\\
Dovrà essere presente un indicatore dei minuti attivi settimanali a confronto con i minuti consigliati dall'OMS, la lista degli obiettivi attualmente attivi con la relativa scadenza e il progresso e la lista degli appuntamenti in arrivo.\\
Questa pagina deve essere facilmente scalabile con l'aggiunta di ulteriori informazioni.
\subsection{Move}
In questa pagina l'utente dovrà poter vedere tutti i suoi obiettivi legati al movimento.\\
Dovranno quindi essere presenti tutti gli obiettivi attivi e passati, per ogni obiettivo attivo dovrà potersi vedere il dettaglio e lo stato attuale, mentre per i passati il cliente dovrà poterli filtrare per data e per completamento, così da capire dove può migliorare e in che periodo è stato più produttivo.\\
Dovrà inoltre essere presente una schermata in cui il cliente possa vedere tutti gli \textit{achievement} relativi al movimento da lui sbloccati come se fosse una sorta di medagliere, quest'ultima opzione è stata pensata in ottica \gls{gamification}. 
\subsection{Food}
Questa pagina non è oggetto del mio progetto di tirocinio.\\
Dovranno essere presenti tutti gli obiettivi attivi e passati legati allo stato di alimentazione del cliente.\\
\subsection{Health}
Questa pagina non è oggetto del mio progetto di tirocinio.\\
Dovranno essere presenti degli indicatori sullo stato di salute del cliente, come l'andamento della frequenza cardiaca nell'arco della giornata.
\subsection{Agenda}
In questa pagina l'utente dovrà avere un resoconto di tutti i suoi obiettivi e appuntamenti nel calendario. Sarà quindi presente un calendario in cui per ogni giorno l'utente dovrà vedere i suoi appuntamenti e i suoi obiettivi.
\subsection{Chat}
Questa pagina non è oggetto del mio progetto di tirocinio a livello implementativo, mi sono occupato principalmente della progettazione.
In questa pagina l'utente può comunicare con il suo coach.\\
Sono stati analizzati diversi servizi ed \gls{SDK} per l'implementazione di un sistema di messaggistica, che in futuro dovrà anche gestire delle video chiamate.\\
Tra tutti i servizi analizzati si è pensato di utilizzare Google Firebase, in quanto rende possibile anche l'utilizzo delle notifiche push che possono tornare utili per altri scopi.
Qui il cliente avrà sempre aperta la chat con il suo coach, al quale potrà inviare messaggi testuali, foto, video o messaggi vocali. Per tutelare cliente e coach non si utilizzerà in alcun modo il numero di cellulare, la chat verrà gestite interamente da Google Firebase e dal nostro backend.\\
Inoltre per evitare che il coach si trovi troppi messaggi da tutti i suoi clienti, ogni cliente avrà una sorta di piano di abbonamento. Con l'abbonamento base gratuito avrà a disposizione un numero limitato di messaggi e minuti di chiamate al mese, mentre acquistando un abbonamento premium potrà aumentare il numero di messaggi e di minuti di chiamate.
%*************************************************************
\section{Applicazione coach}
La progettazione dell'applicazione coach segue la progettazione dell'applicazione client. Infatti viste le somiglianze e le funzionalità in comune tra le due applicazioni, si è deciso di utilizzare la stessa \gls{codebase}, e di sfruttare i flavors per compilare due diverse applicazioni a partire dallo stesso sorgente.\\
I flavors permettono di definire configurazioni di build separate che, tramite parametri e variabili d’ambiente, ci consentono di costruire l’applicazione distinguendo quali moduli fanno parte dell’applicazione coach e quali client. In questo modo possiamo inoltre selezionare icone diverse, indicare URL diversi per le API coach e client e generare \textit{appid} differenti per poter pubblicare negli store due applicativi differenti.
