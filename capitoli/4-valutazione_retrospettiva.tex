% !TEX encoding = UTF-8
% !TEX TS-program = pdflatex
% !TEX root = ../tesi.tex

%**************************************************************
\chapter{Valutazione Retrospettiva}
\label{cap:valutazione-retrospettiva}
%**************************************************************

\section{Difficoltà incontrate}
\label{sec:difficolta_incontrate}
Le prime difficoltà incontrate a livello personale sono state di tipo tecnologico: non mi ero mai trovato a programmare con linguaggi orientati al \textit{frontend}, e ho dovuto quindi cambiare almeno in parte il mio \textit{modus operandi} nella progettazione e scrittura di codice. A questo si è aggiunta la necessità di gestire altri tre linguaggi per le implementazioni native, ovvero Java e Kotlin per il lato Android e Swift per la parte iOS e, a complicare le cose, tutti questi linguaggi dovevano essere messi in comunicazione diretta con Flutter (e nel caso di Java e Kotlin anche tra di loro).\\ 
Una buona parte dello \textit{stage} è quindi stata investita nel comprendere e nel gestire la grande diversità di linguaggi diversi.

\begin{figure}[H]
  \centering
  %\includegraphics[height=5cm]{screen_mobilesyn}
  \includegraphics[width=.5\textwidth]{asa_google_search}\hfill
  \includegraphics[width=.5\textwidth]{flutter_google_search}\\
  \includegraphics[width=1\textwidth]{asa_flutter_google_search}
  \caption[Ricerca esatta Flutter e ASA 23 novembre]{Al 23 novembre 2022, una ricerca esatta di "flutter" mostra 87.5 milioni di risultati rilevanti, di "azure spatial anchors" 41 milioni mentre la combinazione delle due, solo 17}
\label{fig:search1}
\end{figure}

In linea più generale invece le due problematiche che hanno più di tutte complicato i lavori (sia per me che per Datasoil stessa) sono stati la mancanza di documentazione adeguata e la mancanza di supporto da parte della comunità degli sviluppatori.\\
Microsoft non fornisce un \textit{set} documentale adeguato per quanto riguarda \asa{}, mostrando il meno possibile della struttura interna (ad esempio come viene rappresentato un ancoraggio) e fornendo solo \api{} per effettuare operazioni ad alto livello (come il salvataggio in \textit{cloud} di un'\textit{anchor}), inoltre la ricerca della documentazione è macchinosa.\\
L'altro problema risiede nella difficoltà estrema di trovare supporto di terze parti (ad esempio in siti come \url{https://stackoverflow.com}), come viene mostrato nelle figure \ref{fig:search1} e \ref{fig:search2}.

\begin{figure}[H]
  \centering
  %\includegraphics[height=5cm]{screen_mobilesyn}
  \includegraphics[width=.8\textwidth]{asa_flutter_search_today}
  \caption[Ricerca esatta Flutter e ASA 8 febbraio]{All'otto febbraio 2023 non si notano miglioramenti: solo 14 risultati rilevanti, alcuni in lingua giapponese, mostrano quando sia difficile trovare informazioni sull'argomento.}
  \label{fig:search2}
\end{figure}

Questa situazione ha significato dover risolvere tutti i problemi incontrati senza poter fare affidamento su alcun aiuto esterno e ha quindi esacerbato ulteriormente l'approccio \textit{trial-and-error} di cui si è parlato nella sezione \ref{sec:pianificazione}.

%************************************************************************************************************

\section{Soddisfacimento obiettivi}
\subsection{Valutazione personale}
Riprendendo quanto detto in sezione \ref{sec:motivazione_scelta} traggo un bilancio sul mio grado di soddisfacimento personale rispetto allo \textit{stage} svolto.

\begin{itemize}
  \item Acquisire competenze di programmazione \textit{frontend}:
      \begin{itemize}
          \item Utilizzare linguaggi specifici (ad esempio Dart);
          \item Sviluppare componenti grafiche per applicazioni \textit{mobile};
          \item Sviluppare familiarità con strumenti comunemente usati (ad esempio \vsc);
      \end{itemize}
\end{itemize}

Questo primo obiettivo è stato soddisfatto quasi nella sua interezza: ho avuto modo di sviluppare con Dart e ho usato estensivamente \vsc{} e \astudio{}, tuttavia il lavoro nel suo complesso è stato più incentrato sull'infrastruttura sottostante piuttosto che la creazione di molteplici elementi grafici e quindi la sensazione di aver "creato un'applicazione", o meglio di aver lavorato sul \textit{fronted} piuttosto che sul \textit{backend}, è venuta meno.

\begin{itemize}
  \item Osservare direttamente il lavoro in azienda \textit{software}:
    \begin{itemize}
        \item Vedere organizzazione giornaliera lavori;
        \item Valutare tempistiche e ritmi lavorativi;
        \item Valutare equilibrio vita-lavoro;
    \end{itemize}
\end{itemize}

Il secondo obiettivo è stato completamente soddisfatto: complice un'organizzazione settimanale che ha compreso tre giorni in presenza e tre giorni di lavoro a distanza ho potuto farmi un'idea nel complesso positiva di quello che è l'equilibrio vita-lavoro in ambiente aziendale (o perlomeno in ambiente di \textit{startup}).\\
Ho potuto vedere i ritmi lavorativi all'interno dell'ufficio e farmi un'idea molto più chiara di cosa aspettarmi una volta uscito dal cammino universitario.

\begin{itemize}
  \item Acquisire competenze di sviluppo \textit{mobile}:
        \begin{itemize}
            \item Sviluppare familiarità con strumenti comunemente usati (ad esempio emulatori);
            \item Sviluppare componenti applicazione \textit{mobile};
            \item Utilizzare \textit{framework} specifici (ad esempio Flutter);
        \end{itemize}
\end{itemize}

Il terzo obiettivo è stato completamente soddisfatto: ho avuto modo di sviluppare e provare applicazioni costruite in Flutter sia su emulatore che sul mio telefono personale, usando strumenti specifici com \astudio{} o i vari \textit{plugin} di Flutter per \vsc{}.

\begin{itemize}
  \item Acquisire competenze di realtà aumentata:
        \begin{itemize}
            \item Comprendere teoria dietro concetti come ancoraggi e tridimensionalità;
            \item Conoscere tecnologie principali usate nel campo;
            \item Valutare se sia di mio interesse perseguire studi futuri sull'argomento.
        \end{itemize}
\end{itemize}

Il quarto obiettivo è stato soddisfatto a un livello che ritengo soddisfacente: ho avuto modo di usare un sottoinsieme rilevante delle tecnologie maggiori (ARCore, ARKit e \asa{}) e ne ho studiato la teoria sottostante. Sebbene abbia ancora qualche riserva sull'uso di queste risorse in ambito professionale, ritengo abbiano un grande potenziale.\\
Nel complesso ritengo questa esperienza di \textit{stage} più che positiva e arricchente da un punto di vista sia tecnico che professionale.

\subsection{Valutazione proponente}
Riprendendo quanto riportato in sezione \ref{sec:obiettivi} traggo un bilancio sul grado di soddisfacimento dell'azienda nei confronti dei risultati ottenuti nel corso dello \textit{stage}.
\aCapo{}
\textbf{Obiettivi minimi:}

\begin{itemize}
  \item Studio e comprensione del linguaggio di programmazione Dart e del \textit{framework} Flutter;
\end{itemize}

Requisito soddisfatto, come si può notare dall'appendice \ref{ch:flutter}.

\begin{itemize}
  \item Analisi strumenti Azure: Azure Console e \asa{};
\end{itemize}

Requisito soddisfatto, come si può notare nella sezione \ref{subsec:azure}.

\begin{itemize}
  \item Ricerca di \textit{framework} e \sdk{} per implementare le \asa{} in Flutter;
\end{itemize}

Requisito soddisfatto, come si può notare nella sezione \ref{subsec:framework_ar}.

\begin{itemize}
  \item Eventuale studio di linguaggi necessari alle implementazioni native, come ad esempio Kotlin, Java, Unity o Swift;
\end{itemize}

Requisito soddisfatto, come si può notare nei frammenti \ref{lst:android_channels} e \ref{lst:ios_channels}.

\begin{itemize}
  \item Completamento del \textit{framework} scelto nell'app Flutter esistente;
\end{itemize}

Requisito soddisfatto, come si può notare nei frammenti \ref{lst:mobilesyn_managers} e \ref{lst:arplug_manager}.

\begin{itemize}
  \item Completamento dello sviluppo dei componenti per rappresentare le \textit{anchor} nello spazio in realtà aumentata;
\end{itemize}

Requisito soddisfatto, come si può notare nel frammento \ref{lst:ar_view}.

\begin{itemize}
  \item Completamento dello sviluppo delle \api{} per l'interazione utente con gli ancoraggi in realtà aumentata;
\end{itemize}

Requisito soddisfatto, come si può notare nei frammenti \ref{lst:mobilesyn_asset_provider}, \ref{lst:mobilesyn_ticket_provider} e \ref{lst:mobilesyn_asset_ticket_provider}.
\aCapo{}
\textbf{Obiettivi massimi:}

\begin{itemize}
  \item Sviluppo di componenti per mappare \textit{asset} aziendali ad \textit{anchor};
\end{itemize}

Requisito soddisfatto, come si può notare in figura \ref{fig:place_asset}.

\begin{itemize}
  \item Sviluppo di componenti grafici per la visualizzazione di metriche e informazioni di controllo.
\end{itemize}

Requisito non soddisfatto.\aCapo{}
Nel complesso gli obiettivi posti dal proponente sono stati raggiunti con un soddisfacente grado di completamento, rendendo quindi positiva la valutazione generale.

\section{Competenze}
\subsection{Competenze acquisite}
\subsection{Competenze mancanti}
\subsubsection{Lacune corso di studi}
