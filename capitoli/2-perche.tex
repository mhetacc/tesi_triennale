% !TEX encoding = UTF-8
% !TEX TS-program = pdflatex
% !TEX root = ../tesi.tex

%**************************************************************
\chapter{Lo Stage}
\label{cap:lo-stage}
%**************************************************************
\section{Descrizione e temi}
Il progetto proposto prevede l'implementazione di una vista in realtà aumentata per l'applicazione di ticketing e monitoraggio aziendale "\textbf{SYN}".\\
La componente AR dovrà necessariamente appoggiarsi al servizio di {asa} perchè gli {asset} aziendali tracciati da \textbf{SYN} hanno un {digital twin} ancorato e localizzato proprio tramite le {anchor} di Microsoft, e dovrà idealmente essere cross-platform quindi fornendo l'implementazione adeguata sia per Android che per iOS.\\
Dalla vista dovrà essere possibile interagire con gli {asset}, quindi ottenendoli dal cloud e presentandoli nella scena AR, e con i ticket a loro associati, provocando l'apparizione di una card all' on-tap dell'{asset}, ovvero mostrando a schermo tramite una lista testuale i vari ticket relativi all'{asset} dopo che questo viene toccato.\\
Per concludere l'interezza degli obiettivi sarà necessario studiare dei {framework} o degli {SDK} che permettono l'integrazione di {asa} in codice nativo (che sarà Java per Android e Swift per iOS) con il {framework} Flutter e il linguaggio su cui si poggia, ovvero Dart.

\section{Strategia aziendale}
\section{Attività a supporto}


\section{Obiettivi}
\begin{itemize}
    \item Minimi:
        \begin{itemize}
            \item Studio e comprensione del linguaggio di programmazione \textit    {Dart} e del {framework} \textit{Flutter};
            \item Analisi toolkit Azure: Azure Console e {asa};
            \item Ricerca di {framework} e {SDK} per implementare le    {asa} in \textit{Flutter};
            \item Eventuale studio di linguaggi ulteriori necessari alle    implementazioni native, come ad esempio \textit{Kotlin}, \textit{Java},    \textit{Unity} o \textit{Swift};
            \item Completamento del {framework} scelto nell'app \textit{Flutter}    esistente;
            \item Completamento dello sviluppo dei componenti per rappresentare le {anchor} nello spazio AR;
            \item Completamento dello sviluppo delle \acrshort{api} per l'interazione utente con le {anchor} AR;
        \end{itemize}
    \item Massimi: 
        \begin{itemize}
            \item Sviluppo di componenti per mappare {asset} aziendali ad {anchor} AR;
            \item Sviluppo di componenti UI per la visualizzazione di metriche e informazioni di controllo.
        \end{itemize}
\end{itemize}

\subsection{Relazione con strategia aziendale}
\section{Vincoli}
\section{Motivazione scelta}
