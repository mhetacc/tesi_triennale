% !TEX encoding = UTF-8
% !TEX TS-program = pdflatex
% !TEX root = ../tesi.tex

%**************************************************************
\chapter{Analisi dei requisiti}
\label{cap:analisi-requisiti}
%**************************************************************

\section{Requisiti individuati}

Di seguito sono riportati i requisti individuati nel piano di lavoro proposto e a seguito di opportuni confronti con il tutor aziendale. Essi sono catalogati secondo la dicitura:
\begin{center}
    \textbf{R[Obbligatorietà][Tipologia][Codice]}
\end{center}
dove:
\begin{itemize}
    \item \textbf{Obbligatorietà}: specifica quanto un requisito sia vincolante per la riuscita del prodotto e può assumere i seguenti valori:
    \begin{itemize}
        \item \textbf{1: } Requisito obbligatorio;
        \item \textbf{2: } Requisito desiderabile ma non essenziale per il funzionamento;
        \item \textbf{3: } Requisito opzionale.
    \end{itemize}
    \item \textbf{Tipologia}: specifica la tipologia del requisito e può assumere i seguenti valori:
    \begin{itemize}
        \item \textbf{F: }\textit{funzionale,} determina una funzionalità necessaria all'applicazione;
        \item \textbf{V: }\textit{vincolo,} riguarda una caratteristica del prodotto decisa a monte.
    \end{itemize}
    \item \textbf{Codice}: identifica univocamente un requisito all'interno della sua tipologia (ovvero possono esistere due requisiti con lo stesso codice a patto che siano uno funzionale e uno di vincolo). Per i requisti subordinati si usa il "punto" come divisorio (\textit{ReqPadre, ReqPadre.Figlio1})
\end{itemize}

\subsection{Requisiti funzionali}

{
    \setlength{\freewidth}{\dimexpr\textwidth-10\tabcolsep}
    \renewcommand{\arraystretch}{1.5}
    \centering
    \setlength{\aboverulesep}{0pt}
    \setlength{\belowrulesep}{0pt}
    \rowcolors{2}{red!10}{white}
    \begin{longtable}{C{.15\freewidth} | C{1\freewidth}}
       \toprule
    \rowcolor{red}
    \textcolor{white}{\textbf{Codice}}&
    \textcolor{white}{\textbf{Descrizione}}\\
    \toprule
    \endhead
    
\iffalse
    %INTEGRAZIONE FRAMEWORK SCELTO NELL'APP FLUTTER ESISTENTE
    &\\
    %SVILUPPO COMPONENTI PER RAPPRESENTARE ANCHOR IN AR
    &\\
    %SVILUPPO API PER INTERAZIONE UTENTE CON ANCHOR AR
    &\\
    %SVILUPPO COMPONENTI PER MAPPARE ASSET AZIENDALI AD ANCHOR AR
    &\\
    %SVILUPPO COMPONENTI UI PER VISUALIZZARE METRICHE ED INFO DI CONTROLLO
    &\\
\fi
    %AR
    R1F1 & Il plugin-in deve rappresentare asset tramite Anchor AR\\
    R1F2 & Il plugin-in deve rappresentare ticket tramite Anchor AR\\
    R1F3 & Il plug-in deve integrare le Anchor tramite asa\\
    R1F3.1 & Permettere aggiunta di asa\\%C
    R1F3.2 & Permettere recupero e visualizzazione di asa\\%R
    R2F3.3 & Permettere modifica di asa\\%U
    R1F3.4 & Permettere eliminazione di asa\\%D
    %API BACKEND
    R1F4 & Comunicare con Syn API\\
    R1F4.1 & Ricevere asset con Anchor associata\\
    R1F4.2 & Aggiungere asset con Anchor associata\\
    R2F4.3 & Ricevere ticket con Anchor associata\\
    R2F4.4 & Aggiungere ticket con Anchor associata\\
    %UI FLUTTER
    R1F5 & Utente deve poter vedere quali asset hanno anchor associata\\
    R2F6 & Utente deve poter vedere quali ticket hanno anchor associata\\
    R1F7 & Utente deve poter raggiungere l'anchor in vista AR dalla schermata dell'asset\\
    R1F8 & On-Tap su una anchor deve aprire una bottom sheet contestuale\\
    R1F8.1 & Bottom sheet deve presentare identificatore per asset o ticket associato alla anchor\\
    R1F8.2 & Bottom sheet associato a un asset mostra ultimi tre ticket aperti\\
    R1F8.3 & Bottom sheet deve fornire Call-To-Action per eliminare l'anchor\\
    R1F8.4 & Bottom sheet deve fornire Call-To-Action per raggiungere pagina di dettaglio\\
    R1F9 & Le informazioni contestuali di un ticket includono data e ora di creazione\\
    %UI AR
    R1F10 & Anchor hanno rappresentazione visiva contestuale\\ 
    R1F10.1 & Asset vengono rappresentati come \todo\\
    R1F10.2 & Ticket vengono rappresentati come \todo\\
    R1F11 & On-Tap sullo spazio permette di creare una Anchor in posizione\\
    R1F11.1 & Anchor posizionata nello spazio può essere salvata\\
    R1F11.2 & Anchor posizionata nello spazio può essere eliminata\\
    R1F12 & Il salvataggio di un'Anchor è disponibile solo quando è sicuro vada a buon fine\\
    R1F12.1 & Viene mostrato a schermo un feedback riguardo il livello di sicurezza raggiunto\\
    \bottomrule
    \rowcolor{white} 
    \caption{Tabella dei requisiti funzionali}
    \label{tab:requisiti-funzionali}
    \end{longtable}
}



\subsection{Requisiti di vincolo}

{
    \setlength{\freewidth}{\dimexpr\textwidth-10\tabcolsep}
    \renewcommand{\arraystretch}{1.5}
    \centering
    \setlength{\aboverulesep}{0pt}
    \setlength{\belowrulesep}{0pt}
    \rowcolors{2}{red!10}{white}
    \begin{longtable}{C{.15\freewidth} C{1\freewidth}} 
       \toprule
    \rowcolor{red}
    \textcolor{white}{\textbf{Codice}}&
    \textcolor{white}{\textbf{Descrizione}}\\
    \toprule
    \endhead

    R1V1 & Framework scelto funziona su Android\\
    R2V2 & Framework scelto funziona su iOS\\
    R1V3 & Framework scelto si integra con API Syn\\
    R1V3.1 & Framework ottiene con Anchor associata i dati degli asset\\
    R1V3.2 & Framework ottiene con Anchor associata i dati dei ticket\\
    R1V4 & Il framework scelto utilizza asa\\
    R1V5 & La vista AR deve essere sviluppata in Flutter\\
    \bottomrule
    \rowcolor{white} 
    \caption{Tabella dei requisiti di vincolo}
    \label{tab:requisiti-di-vincolo}
    \end{longtable}
}



\section{Riepilogo requisiti}
\subsection{Riepilogo requisiti app cliente}
Sono stati individuati un totale di 36 requisiti, 29 funzionali e 7 di vincolo, di seguito schematizzati:

{
    \setlength{\freewidth}{\dimexpr\textwidth-10\tabcolsep}
    \renewcommand{\arraystretch}{1.5}
    \centering
    \setlength{\aboverulesep}{0pt}
    \setlength{\belowrulesep}{0pt}
    \rowcolors{2}{red!10}{white}
    \begin{longtable}{C{.25\freewidth} C{.2\freewidth}} 
       \toprule
    \rowcolor{red}
    \textcolor{white}{\textbf{Obbligatorietà}}&
    \textcolor{white}{\textbf{Quantità}}\\
    \toprule
    \endhead

    Obbligatori & 31\\
    Desiderabili & 5\\
    \bottomrule
    \rowcolor{white} 
    \caption{Numero di requisiti per obbligatorietà}
    \label{tab:requisiti-obbligatorieta}
    \end{longtable}
}

{
    \setlength{\freewidth}{\dimexpr\textwidth-10\tabcolsep}
    \renewcommand{\arraystretch}{1.5}
    \centering
    \setlength{\aboverulesep}{0pt}
    \setlength{\belowrulesep}{0pt}
    \rowcolors{2}{red!10}{white}
    \begin{longtable}{C{.25\freewidth} C{.2\freewidth}} 
       \toprule
    \rowcolor{red}
    \textcolor{white}{\textbf{Tipologia}}&
    \textcolor{white}{\textbf{Quantità}}\\
    \toprule
    \endhead

    Funzionali & 29\\
    Di Vincolo & 7\\
    \bottomrule
    \rowcolor{white} 
    \caption{Numero di requisiti per tipologia}
    \label{tab:requisiti-tipolgia}
    \end{longtable}
}
