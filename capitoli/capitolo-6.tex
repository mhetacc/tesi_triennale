% !TEX encoding = UTF-8
% !TEX TS-program = pdflatex
% !TEX root = ../tesi.tex

%**************************************************************
\chapter{Conclusioni}
\label{cap:conclusioni}
%**************************************************************
In questo capitolo vengono analizzati i risultati ottenuti durante l'attività di stage e vengono esposte le conclusioni tratte.
%**************************************************************
\section{Valutazione dei risultati ottenuti}
Il prodotto finale ottenuto rispetta le aspettative dell'azienda. L'interfaccia grafica risulta sobria e \textit{user-friendly}, inoltre grazie alla buona gestione dello stato dell'applicazione i rendering dei widget sono ben gestiti e non sono presenti rallentamenti nell'utilizzo dell'app.\\
L'app client e l'app coach rispettano tutti i requisiti funzionali obbligatori e desiderevoli delineati nell' Analisi dei Requisiti. Per quanto riguarda invece i requisiti di vincolo, per entrambe le app non è stato rispettato il requisito desiderabile \textbf{R2V3}: "L'applicazione deve funzionare completamente su dispositivi iOS". Questo è dovuto al poco tempo disponibile dedicato allo sviluppo per iOS, si è preferito infatti completare lo sviluppo e il testing per Android.
%**************************************************************
\section{Sviluppi futuri}
L'applicazione client una volta conclusa dovrà comporsi di tre diverse sezioni principali: Move, Food e Health.\\
Il mio stage ha coperto principalmente la sezione Move, predisponendo solamente delle schermate vuote per le altre due rimanenti.\\
I prossimi sviluppi saranno dunque la sezione di Food: sezione in cui il cliente avrà degli obiettivi alimentari da raggiungere e potrà vedere la sua cronistoria alimentare; e la sezione Health in cui il cliente potrà monitorare il suo stato di salute attraverso diversi grafici che andranno ad illustrare l'andamento del suo battito cardiaco nell'arco delle giornate e delle attività motorie.\\
Altra miglioria futura dovrà essere il \textit{restyling} grafico di alcuni elementi dell'applicazione che per mancanza di tempo si è deciso di lasciare solamente abbozzati.\\
Allo stesso modo anche per l'applicazione coach dovranno essere sviluppati degli elementi per raffigurare gli obiettivi di alimentazione e lo stato di salute di un cliente, per questa applicazione però non si dovranno produrre delle intere sezioni ma solamente alcuni elementi riassuntivi contenenti tutti i dati alimentari e di salute del cliente.\\
Invece per entrambe le applicazioni una funzionalità da sviluppare che andrà a dare grande valore al progetto è il sistema di messaggistica attualmente abbozzato. L'idea è quella di sviluppare un \gls{bot} che risponda automaticamente alle domande che più frequentemente vengono fatte dai clienti, mentre la chat diretta tra coach e cliente viene aperta solamente se il \gls{bot} non sa dare risposta ai dubbi del cliente.\\
Chiaramente il cliente non potrà abusare di questo sistema e avrà quindi un numero limitato di messaggi da poter inviare gratuitamente, al raggiungimento della soglia massima il cliente potrà sottoscrivere un abbonamento per aumentare il numero di messaggi inviabili.\\
Ultimo punto di estensione è la possibilità di effettuare delle video chiamate tra cliente e coach. Per quest'ultimo punto non è stata effettuata nessun tipo di progettazione ma l'idea è piaciuta all'azienda e in futuro verrà implementata.
%**************************************************************
\section{Conoscenze acquisite}
Durante il percorso di stage ho potuto approfondire le mie conoscenze di sviluppo mobile e di conoscere il framework Flutter.
Prima dello stage avevo già avuto l'occasione di sviluppare per dispositivi mobile e di utilizzare il framework React per la costruzione di interfacce grafiche web. Sicuramente la mia esperienza mi ha aiutato nell'iniziare il progetto e il percorso di stage mi ha permesso di migliorare nell'ambito dello sviluppo frontend per dispositivi mobile.\\
In particolare partecipando attivamente alla fase di progettazione e di realizzazione del prodotto ho imparato come si può strutturare un progetto complesso e che durante la realizzazione è importante fare attenzione a come le proprie scelte incidano sulla user experience del prodotto finale.\\
Dal punto di vista dell’integrazione all’interno dell’azienda, ho potuto acquisire nozioni sull’organizzazione e sulle fasi del ciclo di vita di un software, oltre alla possibilità di capire come relazionarsi all’interno di un team in ambito lavorativo.
%**************************************************************
\section{Utilizzo di Flutter}
L'esperienza con Flutter è stata positiva, infatti in seguito a qualche settimana di apprendimento mi è stato possibile costruire un'applicazione complessa in maniera semplice ed intuitiva. Inizialmente ho dedicato molto tempo al capire come funzionasse la gestione dello stato e delle chiamate asincrone, questa è stata la parte sicuramente più complessa nello studio del framework.\\ Mentre per la parte di gestione dei widget e di costruzione dell'interfaccia utente Flutter è molto semplice, grazie anche alla mia esperienza con React sono riuscito fin da subito a costruire schermate in modo intuitivo.\\

%**************************************************************
\section{Valutazione personale}
Questo stage è stato una buona esperienza formativa, ho avuto la possibilità di crescere a livello lavorativo sia acquisendo nuove competenze ma anche imparando a lavorare in un team nel quale è sempre stato possibile dialogare e scambiare idee.\\
Durante questa esperienza mi sono trovato spesso davanti a compiti che inizialmente non sapevo come affrontare ma che alla fine ho sempre portato a termine, penso dunque che le competenze fornite dalla laurea triennale siano una buona base dalla quale partire per studiare la maggior parte delle tecnologie usate in ambito lavorativo.