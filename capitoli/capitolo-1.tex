% !TEX encoding = UTF-8
% !TEX TS-program = pdflatex
% !TEX root = ../tesi.tex

%**************************************************************
\chapter{Introduzione}
\label{cap:introduzione}
%**************************************************************

\section{L'azienda}
\begin{figure}[ht]
    \centering
    \includegraphics[height=3cm]{ds_logo}
    \caption{Logo Datasoil S.r.l.}
\end{figure} 
\aCapo{}
Datasoil S.r.l. è una startup innovativa che si occupa di sviluppare applicativi dedicati alla gestione aziendale, altamente integrati con industria 4.0 e Digital Twin: vengono impiegati machine learning e analisi predittive per ottenere analitiche migliori rispetto alla concorrenza, con un occhio attento a innovazione, sicurezza e scalabilità, ottenute tramite prototipazione, MVP e Infrastrucutre as Code, delegando quindi la gestione delle componenti Server a servizi terzi come AWS.\\
Il loro obiettivo è manipolare ed elaborare dati di natura inerentemente caotica, ordinarli e presentarli all'utente finale con UI intuitive e interattive, promuovendo un'interazione efficace tra utilizzatore e mezzo. Prediligono un approccio ai problemi condiviso e modulare per far nascere soluzioni inedite da applicare a grandi progetti, forti di un know-how che abbraccia esigenze verticali.\\
Le piattaforme dedicate ad industry 4.0, smart building e smart city poggiano su dei data silos attualmente esistenti, garantendo un'interpretazione integrata e di alto livello delle informazioni aziendali fondendole con eventi provenienti dai diversi business level creando insight proattivi in tempo reale. I sistemi predittivi di Datasoil sono in grado di notificare la persona giusta al momento giusto anticipando la gestione delle criticità interfunzionali.

%**************************************************************
\section{Il progetto e lo stage}

\subsection{Descrizione}
Il progetto proposto prevede l'implementazione di una vista in realtà aumentata per l'applicazione di ticketing e monitoraggio aziendale "\textbf{SYN}".\\
La componente AR dovrà necessariamente appoggiarsi al servizio di \gls{asa} perchè gli \gls{asset} aziendali tracciati da \textbf{SYN} hanno un \gls{digital twin} ancorato e localizzato proprio tramite le \gls{anchor} di Microsoft, e dovrà idealmente essere cross-platform quindi fornendo l'implementazione adeguata sia per Android che per iOS.\\
Dalla vista dovrà essere possibile interagire con gli \gls{asset}, quindi ottenendoli dal cloud e presentandoli nella scena AR, e con i ticket a loro associati, provocando l'apparizione di una card all' on-tap dell'\gls{asset}, ovvero mostrando a schermo tramite una lista testuale i vari ticket relativi all'\gls{asset} dopo che questo viene toccato.\\
Per concludere l'interezza degli obiettivi sarà necessario studiare dei \gls{framework} o degli \gls{SDK} che permettono l'integrazione di \gls{asa} in codice nativo (che sarà Java per Android e Swift per iOS) con il \gls{framework} Flutter e il linguaggio su cui si poggia, ovvero Dart.


\subsection{Obiettivi}
\begin{itemize}
    \item Minimi:
        \begin{itemize}
            \item Studio e comprensione del linguaggio di programmazione \textit    {Dart} e del \gls{framework} \textit{Flutter};
            \item Analisi toolkit Azure: Azure Console e \gls{asa};
            \item Ricerca di \gls{framework} e \gls{SDK} per implementare le \gls   {asa} in \textit{Flutter};
            \item Eventuale studio di linguaggi ulteriori necessari alle    implementazioni native, come ad esempio \textit{Kotlin}, \textit{Java},    \textit{Unity} o \textit{Swift};
            \item Completamento del \gls{framework} scelto nell'app \textit{Flutter}    esistente;
            \item Completamento dello sviluppo dei componenti per rappresentare le \Gls{anchor} nello spazio AR;
            \item Completamento dello sviluppo delle \acrshort{api} per l'interazione utente con le \Gls{anchor} AR;
        \end{itemize}
    \item Massimi: 
        \begin{itemize}
            \item Sviluppo di componenti per mappare \gls{asset} aziendali ad \Gls{anchor} AR;
            \item Sviluppo di componenti UI per la visualizzazione di metriche e informazioni di controllo.
        \end{itemize}
\end{itemize}



\subsection{Pianificazione del lavoro}
\todo probabile necessità di modifica, ora è più rendicontazione.
\begin{itemize}
    \item \textbf{Settimana 1:} 
        \begin{itemize}
            \item Installazione e configurazione del framework \flutter, della SDK Android 
            \item Configurazione emulatore Android (scelto Pixel 6 API 33) e 
            developers options sul mio telefono personale;
            \item Configurazione \vsc e \astudio;
            \item Sviluppo di piccola app d'esempio con \flutter che sfrutta gli \textit{Stateful Widgets};
        \end{itemize} 
    \item \textbf{Settimana 2:} 
        \begin{itemize}
            \item Studio degli \textit{Hook Widgets}, simili agli \textit{Hooks} di \textit{Rest};
            \item Traduzione degli \textit{Stateful Widgets} dell'app d'esempio in \textit{Hook Widgets};
            \item Sviluppo app d'esempio con gli \textit{Hook Widgets};
        \end{itemize}
    \item \textbf{Settimana 3:} 
        \begin{itemize}
            \item Studio delle asa;
            \item Creazione risorse necessarie nel portale di \textit{Azure};
            \item Deploy app di esempio fornita da \textit{Microsoft};
        \end{itemize}
    \item \textbf{Settimana 4:} Studio dei metodi per implementare le asa in \flutter:
        \begin{itemize}
            \item Studio dei \textit{Method Channels};
            \item Studio di \textit{ARwayKit} e valutazione del suo approccio tramite \unity;
            \item Studio di \arplug;
        \end{itemize}
    \item \textbf{Settimana 5:} 
        \begin{itemize}
            \item Adattamento dell'esempio \arplug per integrarci le asa;
            \item Lettura dei logs per filtrarne gli output;
        \end{itemize}
    \item \textbf{Settimana 6:} Utilizzo diretto di \arcore in \flutter per capirne il workflow;
    \item \textbf{Settimana 7:} Integrazione di asa in \flutter senza errori;
    \item \textbf{Settimana 8:} Integrazione di componenti AR in SYN;
\end{itemize}
%************************************************************************************************************
\section{Prodotto ottenuto}
\todo

\iffalse
Il prodotto ottenuto si compone di due diverse applicazioni: applicazione cliente e applicazione coach.\\
Entrambe le applicazioni sono state sviluppate utilizzando \textit{Flutter} come framework in modo da valutarne anche le potenzialità, infatti l'azienda ha sempre utilizzato React Native per lo sviluppo mobile. Mentre per lo stile dei componenti dell'applicazione si è deciso di utilizzare il \textit{Material Design} di Google già ampiamente utilizzato da Datasoil S.r.l.\\
L'applicazione cliente allo stato attuale permette la visione dei propri obiettivi di movimento e degli appuntamenti, oltre che una schermata di profilo personale con la possibilità di associazione ai diversi profili fitness al fine di rilevare dati.\\
L'app cliente è quindi a circa un terzo dello sviluppo, l'azienda ora potrà continuare con lo sviluppo delle restanti sezioni relative alla dieta e alla visualizzazione dello stato di salute del cliente. \\
L'applicazione coach allo stato attuale permette la visualizzazione della lista dei propri clienti con i relativi obiettivi di movimento e la visualizzazione di un'agenda con gli appuntamenti. L'app coach è in buona parte sviluppata, l'idea aziendale è quella che dall'app il coach possa solo vedere ciò che ha già assegnato dall'interfaccia web (prodotto già sviluppato da Datasoil S.r.l.). \\
L'azienda quindi dovrà solo aggiungere la visualizzazione della dieta e dei parametri di salute del cliente, omettendo la possibilità di assegnare ulteriori obiettivi dall'app mobile. 
Per entrambe le applicazioni è stata predisposta una versione minimale di chat in modo che il cliente possa comunicare con il proprio coach. Questa chat sarà in futuro ampiamente rivista in modo da permettere lo scambio di file multimediali e di video chiamate, oltre che limitare lo scambio di messaggi tramite una sorta di abbonamento.
\fi
%************************************************************************************************************
\section{Difficoltà incontrate}
Le difficoltà primariamente incontrate sono state tre: comprensione dei mezzi da usare, farli comunicare tra loro e la scarsissima assistenza trovata online.\\
Infatti la necessità di familiarizzare in poco tempo con molte componenti (linguaggi, framework, plug-in, SDKs) diverse e mai viste prima, come \flutter, \kotlin, \java, \dart, Android SDK, asa, \arcore e \arplug si è certamente dimostrato complesso, tuttavia non quanto lo è stato far comunicare queste componenti delle quali non ho esperienza tra di loro: ad esempio implementando in \flutter (quindi \dart), sfruttando l'\arplug (che usa \kotlin e \flutter), le asa (scritte in \java) che a loro volta si appoggiano ad \arcore (così come lo fa, in una certa parte, \arplug). \\
Tutto questo è stato esacerbato dalla scarsissima documentazione fornita da \textit{Microsoft} per le asa e dalle pochissime risorse online (compresi forum e issue di \textit{GitHub}) riguardo all'implementazione AR in generale, \textit{ASA} nello specifico, in \flutter.
%**************************************************************
\section{Struttura della relazione}

\begin{description}
    \item[{\hyperref[cap:strumenti-utilizzati]{Il secondo capitolo}}] descrive gli strumenti e le tecnologie utilizzate per svolgere il progetto, particolare attenzione è posta sulla presentazione del framework Flutter, in quanto è alla base del prodotto ottenuto;
    
    \item[{\hyperref[cap:analisi-requisiti]{Il terzo capitolo}}] espone un'analisi tecnica dei requisiti individuati che il prodotto finale deve soddisfare al termine del periodo di stage;
    
    \item[{\hyperref[cap:progettazione]{Il quarto capitolo}}] descrive le scelte progettuali adottate per ottenere una struttura solida manutenibile;
    
    \item[{\hyperref[cap:sviluppo]{Il quinto capitolo}}] espone il percorso di realizzazione del prodotto e le soluzioni alle difficoltà incontrate;
    
    \item[{\hyperref[cap:conclusioni]{Il sesto capitolo}}] descrive l'esperienza di stage e i risultati ottenuti.
\end{description}

Riguardo la stesura del testo, relativamente al documento sono state adottate le seguenti convenzioni tipografiche:
\begin{itemize}
	\item gli acronimi, le abbreviazioni e i termini ambigui o di uso non comune menzionati vengono definiti nel glossario, situato alla fine del presente documento;
	\item i termini in lingua straniera o facenti parti del gergo tecnico sono evidenziati con il carattere \emph{corsivo}.
\end{itemize}
