% !TEX encoding = UTF-8
% !TEX TS-program = pdflatex
% !TEX root = ../tesi.tex

%**************************************************************
\chapter{Strumenti utilizzati}
\label{cap:strumenti-utilizzati}
%**************************************************************

\section{Strumenti di sviluppo}
\subsection{Visual Studio Code}
\begin{figure}[ht]
    \centering
    \includegraphics[height=2cm]{vsc_logo}
    \caption{Logo di Visual Studio Code}
\end{figure} \aCapo{}
\vsc{} è un editor per codice sorgente, sviluppato da \textit{Microsoft} servendosi del framework \textit{Electron}, disponibile per Windows, Linux e macOS. Le sue funzionalità comprendo supporto per debugging, syntax highlighting, intelligent code completion, snippets, code refactoring, e integrazione con Git. \\
Gli utenti possono configurare tema, macro, preferenze e installare estensioni che ne aumentano le funzionalità aggiungendo ad esempio il supporto ai maggiori linguaggi di porgrammazione attualmente presenti.\\
Nella \textit{Stack Overflow 2021 Developer Survey}, \vsc{} è risultato essere l'IDE più popolare, adottato dal 70\% degli intervistati. {\tiny Fonte: \href{https://en.wikipedia.org/wiki/Visual_Studio_Code}{Wikipedia}}\\
\E{} inoltre presente e molto comoda l'estensione di \flutter{} che, tra le altre cose, permette di gestire direttamente in gui i vari dispositivi (smartphone, browser o emulatore) sui quali installare e lanciare l'app che si sta programmando.

\subsection{Android Studio}
\begin{figure}[ht]
    \centering
    \includegraphics[height=2cm]{as_logo}
    \caption{Logo di Android Studio}
\end{figure}\aCapo{}
\textit{Android Studio} è l'IDE ufficiale del sistema operativo di Gooogle, sostituendo Eclipse dal 2015, costruito sul software IntelliJ di JetBrains e progettato specificatamente per lo sviluppo Android. \E{} disponibile per Windows, Linux e macOS.\\
Il 7 maggio 2019 Kotlin rimpiazzò Java come linguaggio consigliato per Android, favorendo un'integrazione ancora maggiore con JetBrains che, appunto, ha sviluppato e prodotto Kotlin stesso. {\tiny Fonte: \href{https://en.wikipedia.org/wiki/Android_Studio}{Wikipedia}}\\
Personalmente ho cercato nonostante tutto di programmare il più possibile su \vsc{} preferendolo quindi ad \textit{Android Studio} in virtù della sua maggiore leggerezza e delle sue più ampie possibilità di personalizzazione.

\subsection{Dart}
\begin{figure}[ht]
    \centering
    \includegraphics[height=2cm]{dart_logo}
    \caption{Logo di Dart}
\end{figure}
Dart è un linguaggio di programmazione sviluppato da Google per il client development, ovvero per web e mobile app, e può anche essere impiegato per la costruzione di applicazione desktop e server.\\
\E{} un linguaggio object-oriented, class-based, garbage-collected, strongly-typed e con una sintassi nello stile di C. Può compilare sia codice macchina che JavaScript


It is an object-oriented, class-based, garbage-collected language with C-style syntax.[10] It can compile to either machine code or JavaScript, and supports interfaces, mixins, abstract classes, reified generics and type inference.[4] 

Dart è un linguaggio di programmazione sviluppato da Google con l'obiettivo di sostituire \textit{javascript} per lo sviluppo web.

\subsection{Flutter}
\begin{figure}[ht]
    \centering
    \includegraphics[height=2cm]{flutter_logo}
    \caption{Logo di Flutter}
\end{figure}
Flutter è un framework open source sviluppato da Google per la costruzione di interfacce grafiche.\\
Flutter usa come linguaggio di programmazione Dart e si compone di quattro diversi componenti principali:
\begin{itemize}
    \item \textbf{Dart platform}: Piattaforma per l'utilizzo del linguaggio Dart, comprende anche la Dart Virtual Machine sulla quale girano le applicazioni Flutter;
    \item \textbf{Flutter engine}: fornisce il supporto per il rendering a basso livello utilizzando la libreria grafica di Google, Skia Graphics. La caratteristica di spicco del Flutter Engine è quella di poter effettuare degli \textit{hot-reload}, ossia visualizzare le modifiche effettuate al codice senza riavviare completamente l'app, ma iniettando il nuovo codice durante lo stato di esecuzione dell'app;
    \item \textbf{Foundation library}: fornisce le classi e le funzioni di base utilizzate per costruire applicazione che utilizzano Flutter ed è scritta in Dart. La creazione di interfacce con Flutter è effettuata per composizione di widget. Ogni widget è renderizzato dal metodo build() che viene chiamato ricorsivamente per ogni figlio, generando così un albero di widget;
    \item \textbf{Design-specific widget}: contiene due set di widget conformi a specifici linguaggi di programmazione. I widget in stile Material Design implementano il design di Google, mentre i widget di Cupertino imitano il design iOS di Apple;
\end{itemize}

Una libreria fondamentale per semplificare l'utilizzo di Flutter è \textbf{Flutter Hooks}. Questa cerca di replicare gli \textit{Hooks} di \textit{React}, ed è utile in quanto si evita di usare \textit{Statefull widget} e \textit{Stateless widget} per la costruzione dei widget. \\
Con Flutter Hooks ogni widget eredita da \textit{Hook widget}, e si può dare uno stato al widget usando l'hook \textit{useState}. Inoltre viene anche più semplice la gestione dei side-effect e del ciclo di vita del widget grazie all'hook \textit{useEffect}.

%**************************************************************
\section{Strumenti organizzativi}

\subsection{Slack}
\begin{figure}[ht]
    \centering
    \includegraphics[height=1.5cm]{slack_logo}
    \caption{Logo di Slack}
\end{figure}

Slack è un software che rientra nella categoria degli strumenti di collaborazione aziendale utilizzato per inviare messaggi in modo istantaneo ai diversi membri del team.\\
Slack è utilizzabile da browser, applicazione desktop e applicazione o mobile.\\
Oltre ai messaggi diretti tra i membri è anche possibile creare diversi canali in modo da separare le conversazioni per argomento, per esempio nel nostro caso utilizzavamo due canali: \#frontend, \#backend.\\
Slack inoltre permette di integrare nel sistema di messaggistica anche altri servizi, come per esempio Github o Jira.

\subsection{Jira}
\begin{figure}[ht]
    \centering
    \includegraphics[height=3cm]{jira_logo.png}
    \caption{Logo di Jira}
\end{figure}

Jira è un software utilizzato per gestire il lavoro collaborativo, segue il principio agile e permette di creare ticket e gestire degli sprint settimanali.\\ Nel nostro caso chiunque poteva aprire un ticket nella sezione \textit{TODO} e assegnarlo a chi di dovere, specificandone i dettagli. Quando il ticket veniva preso in carico lo si spostava nell' apposita sezione \textit{IN PROGRESS} e una volta terminato veniva chiuso nella sezione \textit{DONE}. In questo modo ognuno sa in quale stato è il prodotto e a cosa sta lavorando ogni membro del team.


\subsection{Github}
\begin{figure}[ht]
    \centering
    \includegraphics[height=3cm]{github_logo}
    \caption{Logo di GitHub}
\end{figure}

GitHub è un servizio di hosting per progetti software ed è una implementazione del sistema di versionamento distribuito Git.\\ Il sito è principalmente utilizzato dagli sviluppatori, che caricano il codice sorgente dei loro programmi e lo possono rendere disponibile al resto degli utenti. Questi ultimi possono interagire con lo sviluppatore tramite un sistema di \textit{issue tracking}, \textit{pull request} e commenti che permette di migliorare il codice del \textit{repository} risolvendo bug o aggiungendo funzionalità.
