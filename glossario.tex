
%**************************************************************
% Acronimi
%**************************************************************
\renewcommand{\acronymname}{Acronimi e abbreviazioni}

\newacronym[description={\glslink{apig}{Application Program Interface}}]
    {api}{API}{Application Program Interface}

\newacronym[description={\glslink{umlg}{Unified Modeling Language}}]
    {uml}{UML}{Unified Modeling Language}

\newacronym[description={\glslink{mvpg}{Model View Presenter}}]
    {mvp}{MVP}{Model View Presenter}

%**************************************************************
% Glossario
%**************************************************************
%\renewcommand{\glossaryname}{Glossario}
\renewcommand{\glsnamefont}[1]{\makefirstuc{#1}}


%%%%%%%%%%%%%%%%%%%%%%%%%%%%%%%%%%%%%%%%%%%%%%%%%%%%%%%%%%%%%%%%%%%%%%%%%%%%
\newglossaryentry{asset}
{
    name=asset,
    sort=asset,
    description={todo}
}
\newglossaryentry{digital twin}
{
    name=digital twin,
    sort=digital twin,
    description={todo}
}
\newglossaryentry{anchor}
{
    name=anchor,
    sort=anchor,
    description={todo}
}
\newglossaryentry{asa}
{
    name=Azure Spatial Anchors,
    sort=azure spatial anchors,
    description={todo}
}
\newglossaryentry{branch}
{
    name=branch,
    sort=branch,
    description={Rappresenta una linea di sviluppo indipendente e serve come astrazione per il processo di modifica,
stage o commit.}
}
\newglossaryentry{fork}
{
    name=fork,
    sort=fork,
    description={Sviluppo di un nuovo progetto software che parte dal codice sorgente di un altro già esistente}
}

\newglossaryentry{pull request}
{
    name=pull request,
    sort=pull request,
    description={È una richiesta, fatta all’autore originale di un sofware o di un documento, di includere le nostre personali modifiche al suo progetto. Una pull request controlla automaticamente se sono presenti conflitti tra i file da unire.}
}

\newglossaryentry{agile}
{
    name=agile,
    sort=agile,
    description={È un modello di sviluppo che si contrappone ai classici modelli a cascata o incrementali proponendo un approccio meno strutturato e focalizzato sull'obiettivo di consegnare al cliente, in tempi brevi e frequentemente, software funzionante e di qualità. }
}
\newglossaryentry{BMI}
{
    name=BMI,
    sort=bmi,
    description={Body Mass Index è un dato biometrico, espresso come rapporto tra peso e quadrato dell'altezza di un individuo ed è utilizzato come un indicatore dello stato di peso forma. }
}
\newglossaryentry{bot}
{
    name=bot,
    sort=bot,
    description={Con la parola “bot”, abbreviazione di “robot”, in informatica si intende un programma che ha accesso agli stessi sistemi di comunicazione e interazione con le macchine usate dagli esseri umani.}
}

\newglossaryentry{JWT}
{
    name=JWT,
    sort=jwt,
    description={JSON Web Token è uno standard Internet proposto per la creazione di dati con firma opzionale e/o crittografia opzionale il cui payload contiene un JSON che afferma un certo numero di attestazioni. I token vengono firmati utilizzando una crittografica simmetrica o asimmetrica utilizzando una coppia di chaivi pubblica/privata.}
}
\newglossaryentry{gamification}
{
    name=gamification,
    sort=gamification,
    description={È l'utilizzo di elementi mutuati dai giochi e delle tecniche di game design in contesti non ludici. L'idea è quella di coinvolgere le persone a provare più divertimento e partecipazione nelle attività quotidiane ricche di azioni monotone e noiose attraverso il gioco.}
}
\newglossaryentry{SDK}
{
    name=SDK,
    sort=sdk,
    description={Un Software Developer Kit è un insieme di strumenti per lo sviluppo e la documentazione di software}
}

\newglossaryentry{SRP}
{
    name=SRP,
    sort=srp,
    description={Il Single Responsability Princile afferma che ogni elemento di un programma deve avere una sola responsabilità, e che tale responsabilità debba essere interamente incapsulata dall'elemento stesso. Tutti i servizi offerti dall'elemento dovrebbero essere strettamente allineati a tale responsabilità. }
}
\newglossaryentry{codebase}
{
    name=codebase,
    sort=codebase,
    description={Collezione di codice sorgente usata per costruire una particolare applicazione o un particolare componente.}
}
